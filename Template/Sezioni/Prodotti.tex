%----------------------------------------------------------------------------------------
%	DESCRIPTION OF THE PRODUCTS THAT ARE BEING EXPECTED FROM THE STAGE
%----------------------------------------------------------------------------------------
\section*{Prodotti attesi}
% Personalizzare definendo i prodotti attesi (facoltativo)
Lo studente dovrà produrre una relazione scritta che illustri i seguenti punti.
\begin{enumerate}
    \item Descrizione del progetto \\
    Un riassunto che spieghi di cosa tratta il progetto, una breve
    descrizione dell'azienda, i ruoli che sono stati ricoperti
    durante l'attività di stage. 
    
    \item Tecnologie e strumenti \\
    Una descrizione delle tecnologie utilizzate e degli strumenti di
    supporto (IDE, VCS, ITC ...).
    
    \item Analisi \\
    Descrizione dell'attività di analisi svolta, comprensiva della
    documentazione redatta. Valutazione dei rischi e possibili
    attività di mitigazione.

    \item Progettazione \\
    Descrizione dell'attività di progettazione svolta e delle 
    scelte prese, esplicitando le motivazioni che hanno portato
    a effettuare tali scelte.

    \item Codifica \\
    Descrizione dell'attività di codifica svolta, elencando le
    pratiche di codifica utilizzate. Spiegazione delle parti di
    codice più interessanti per il progetto, con eventuali immagini
    associate se ritenute opportune.

    \item Verifica e validazione \\
    Descrizione dell'attività di verifica e validazione svolta, 
    elencando i test case più importanti per il progetto, associando
    eventuali immagini se ritenute opportune. Spiegazione delle pratiche
    di test utilizzate e spiegazione di ulteriori informazioni interessanti
    per questa attività (percentuale di code coverage ottenuta, percentuale
    di code coverage desiderabile, numero di unit test presenti ...).

    \item Consuntivo finale \\
    Un consuntivo finale, che vada a misurare in termini quantitativi,
    il lavoro svolto durante lo stage, rispetto a quanto pianificato
    inizialmente.

    \item Obiettivi soddisfatti \\
    Un resoconto quantitativo degli obbiettivi e requisiti soddisfatti, 
    rispetto a quelli preventivati.

    \item Resoconto dei rischi \\
    Spegazione se si sono presentati i rischi preventivati e se le misure 
    di mitigazione adottate si sono rivelate efficaci. Eventuali modifiche
    apportate alle misurazioni di mitigazione preventivate e motivazioni 
    che hanno portato alla loro modifica.

    \item Conoscenze acquisite \\
    Descrizione delle conoscenze acquisite, rispetto a prima di svolgere
    l'attività di stage.
\end{enumerate}

Nel qual caso in cui lo studente, in seguito all'analisi, abbia ancora tempo a sua disposizione ... .
