%----------------------------------------------------------------------------------------
%   USEFUL COMMANDS
%----------------------------------------------------------------------------------------

\newcommand{\dipartimento}{Dipartimento di Matematica ``Tullio Levi-Civita''}

%----------------------------------------------------------------------------------------
% 	USER DATA
%----------------------------------------------------------------------------------------

% Data di approvazione del piano da parte del tutor interno; nel formato GG Mese AAAA
% compilare inserendo al posto di GG 2 cifre per il giorno, e al posto di 
% AAAA 4 cifre per l'anno
\newcommand{\dataApprovazione}{Data}

% Dati dello Studente
\newcommand{\nomeStudente}{Andrea}
\newcommand{\cognomeStudente}{Volpe}
\newcommand{\matricolaStudente}{2021904}
\newcommand{\emailStudente}{andrea.volpe.4@studenti.unipd.it}
\newcommand{\telStudente}{+ 39 331 10 97 830}

% Dati del Tutor Aziendale
\newcommand{\nomeTutorAziendale}{Daniele}
\newcommand{\cognomeTutorAziendale}{Zorzi}
\newcommand{\emailTutorAziendale}{d.zorzi@synclab.it}
\newcommand{\telTutorAziendale}{+ 39 380 12 89 060}
\newcommand{\ruoloTutorAziendale}{}

% Dati dell'Azienda
\newcommand{\ragioneSocAzienda}{Sync Lab}
\newcommand{\indirizzoAzienda}{Galleria Spagna, 28, 35127 (PD)}
\newcommand{\sitoAzienda}{https://www.synclab.it/}
\newcommand{\emailAzienda}{info@synclab.it}
\newcommand{\partitaIVAAzienda}{P.IVA 07952560634}

% Dati del Tutor Interno (Docente)
\newcommand{\titoloTutorInterno}{Prof.}
\newcommand{\nomeTutorInterno}{Paolo}
\newcommand{\cognomeTutorInterno}{Baldan}

\newcommand{\prospettoSettimanale}{
     % Personalizzare indicando in lista, i vari task settimana per settimana
     % sostituire a XX il totale ore della settimana
    \begin{itemize}
        \item \textbf{Prima Settimana (40 ore)}
        \begin{itemize}
            \item Incontro con persone coinvolte nel progetto per discutere i requisiti e le richieste relativamente
            al sistema da sviluppare;
            \item Verifica credenziali e strumenti di lavoro assegnati;
            \item Ripasso Java Standard Edition e tool di sviluppo (IDE ecc.);
            \item Studio teorico dell’architettura a microservizi: Passaggio da monolite a MS con pro e contro;
            \item Studio teorico dell’architettura a microservizi: Api Gateway, Service Discovery e Service Regi-
            stry, Circuit Breaker e Saga Pattern;
        \end{itemize}
        \item \textbf{Seconda Settimana - Sottotitolo (40 ore)} 
        \begin{itemize}
            \item Studio Spring Core/Spring Boot;
            \item Studio ORM, in particolare il framework Spring Data/JPA;
        \end{itemize}
        \item \textbf{Terza Settimana - Sottotitolo (40 ore)} 
        \begin{itemize}
            \item Studio Servizi REST e framework Spring DataRest;
            \item Realizzazione di un servizio REST prototipale con Spring;
        \end{itemize}
        \item \textbf{Quarta Settimana - Sottotitolo (40 ore)} 
        \begin{itemize}
            \item Ripasso Javascript;
            \item Studio NodeJS
            \item Studio dei framework Express.JS e Nest.JS;
        \end{itemize}
        \item \textbf{Quinta Settimana - Sottotitolo (40 ore)} 
        \begin{itemize}
            \item Realizzazione di un servizio REST prototipale con NodeJS;
            \item Analisi e Progettazione dei nuovi servizi per il progetto SmartParking;
        \end{itemize}
        \item \textbf{Sesta Settimana - Sottotitolo (40 ore)} 
        \begin{itemize}
            \item Implementazione dei servizi di lettura dati da db;
        \end{itemize}
        \item \textbf{Settima Settimana - Sottotitolo (40 ore)} 
        \begin{itemize}
            \item Implementazione dei servizi di scrittura dati da db;
            \item Confronto tra la soluzione in NodeJS e la soluzione Spring realizzata precedentemente;
        \end{itemize}
        \item \textbf{Ottava Settimana - Conclusione (40 ore)} 
        \begin{itemize}
            \item Termine considerazioni e collaudo finale;
        \end{itemize}
    \end{itemize}
}

% Indicare il totale complessivo (deve essere compreso tra le 300 e le 320 ore)
\newcommand{\totaleOre}{320}

\newcommand{\obiettiviObbligatori}{
	 \item \underline{\textit{O01}}: Acquisizione competenze sulle tematiche sopra descritte;
	 \item \underline{\textit{O02}}: Capacità di raggiungere gli obiettivi richiesti in autonomia seguendo il cronoprogramma;
	 \item \underline{\textit{O03}}: Portare a termine l’implementazione dei microservizi richiesti con una percentuale di
     superamento pari al 80\%;
	 
}

\newcommand{\obiettiviDesiderabili}{
	 \item \underline{\textit{D01}}: Portare a termine l’implementazione dei microservizi richiesti con una percentuale di
     superamento pari al 100\%;
}

\newcommand{\obiettiviFacoltativi}{
	 \item \underline{\textit{F01}}: Studiare come poter implementare le bestPractice dell’architettura a microservizi con NodeJS;
}